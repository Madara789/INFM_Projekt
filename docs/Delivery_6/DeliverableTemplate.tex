% !TeX encoding = UTF-8
% !TeX program = pdflatex
% !BIB program = bibtex

%%% Um einen Artikel auf deutsch zu schreiben, genügt es die Klasse ohne
%%% Parameter zu laden.
\documentclass[]{lni}
\usepackage{graphicx}
\usepackage{dirtree}
\usepackage{listings}
\graphicspath{{./Bilder/}}

% Farben definieren
\definecolor{linkblue}{RGB}{0, 0, 100}
\definecolor{linkblack}{RGB}{0, 0, 0}
\definecolor{comment}{RGB}{63, 127, 95}
\definecolor{darkgreen}{RGB}{14, 144, 102}
\definecolor{darkblue}{RGB}{0,0,168}
\definecolor{darkred}{RGB}{128,0,0}
\definecolor{javadoccomment}{RGB}{0,0,240}
\definecolor{Gray}{RGB}{242,242,242}


% Einstellungen für Quelltexte
\definecolor{backcolour}{rgb}{0.95,0.95,0.92}
\definecolor{codegray}{rgb}{0.5,0.5,0.5}

\lstset{
      xleftmargin=0.1cm,
      basicstyle=\footnotesize\ttfamily,
      keywordstyle=\color{darkgreen},
      identifierstyle=\color{darkblue},
      commentstyle=\color{comment},
      stringstyle=\color{darkred},
      tabsize=2,
      lineskip={2pt},
      columns=flexible,
      inputencoding=utf8,
      captionpos=b,
      backgroundcolor=\color{backcolour},   
      breakautoindent=true,
	  breakindent=2em,
	  breaklines=true,
	  prebreak=,
	  postbreak=,
      numbers=left,                    
      numbersep=5pt,  
      numberstyle=\tiny\color{codegray},  
      showspaces=false,      % Keine Leerzeichensymbole
      showtabs=false,        % Keine Tabsymbole
      showstringspaces=false,% Leerzeichen in Strings
      morecomment=[s][\color{javadoccomment}]{/**}{*/},
      literate={Ö}{{\"O}}1 {Ä}{{\"A}}1 {Ü}{{\"U}}1 {ß}{{\ss}}2 {ü}{{\"u}}1 {ä}{{\"a}}1 {ö}{{\"o}}1
}

\begin{document}
%%% Mehrere Autoren werden durch \and voneinander getrennt.
%%% Die Fußnote enthält die Adresse sowie eine E-Mail-Adresse.
%%% Das optionale Argument (sofern angegeben) wird für die Kopfzeile verwendet.
    \title[Ein Kurztitel]{Subscale Algorythmus}
%%%\subtitle{Untertitel / Subtitle} % if needed
    \author[William Mendat \and Max Ernst \and Steven Schall \and Matthias Reichenbach]
    {William Mendat\footnote{Hochschule Offenburg, Offenburg,
        Deutschland \email{wmendat@stud-hs.offenburg.de}} \and
    Max Ernst\footnote{Hochschule Offenburg, Offenburg,
        Deutschland \email{mernst@stud-hs.offenburg.de}} \and
    Steven Schall\footnote{Hochschule Offenburg, Offenburg,
        Deutschland \email{sschall@stud-hs.offenburg.de}} \and
    Matthias Reichenbach\footnote{Hochschule Offenburg, Offenburg,
        Deutschland \email{mreichen@stud-hs.offenburg.de}}}
    \startpage{1} % Beginn der Seitenzählung für diesen Beitrag / Start page
    \editor{Team Subscale} % Names of Editors
    \booktitle{Subscale Algorythmus} % Name of book title
    \yearofpublication{2022}
%%%\lnidoi{18.18420/provided-by-editor-02} % if known
    \maketitle

    \section{Einleitung}

Anders, als konventionelle Ansätze von Clustering-Algorithmen, die sämtliche Features auf einmal
vergleichen \cite{7022654}, zielt der Subscale Algorithmus darauf ab, hochdimensionalen Daten in Teilen
effizient zu verarbeiten.
Dabei möchte der Algorithmus das durch hohe Dimensionalität bedingte, so genannte Problem \emph{Curse
of Dimensionality} lösen, indem es die Abgeschlossenheit des Apriori-Prinzips
\cite{TechnicalReviewonBuprenorphine:anAlternativeTreatmentforOpioidDependence.1992} nutzt und die
Teilmengen der
Datensatzfeatures sukzessive, bottom-up aufbaut.

Das Apriori-Prinzips ermöglicht es aus der Gesamtmenge der Dimensionen in Teilmengen davon, den
so genannten Subspaces, so zu verarbeiten, dass statt \emph{$2^{k}-1$ möglichen Achsen-parallele
Subspaces in \emph{k} Dimensionen} \cite{7022654}, nur die benötigten Subspaces berechnet werden.

%%% Beginn des Artikeltexts
    \section{Einleitung}
Der Subscale-Algorithmus ist eine effiziente, parallelisier- und folglich auch verteilbare Methode zur Ermittlung von
Clustern in hochdimensionalen Räumen. Ziel des Projektes ist eine Implementierung des Algorithmus, welche sich auf
mehreren geclusterten Maschinen verteilen lässt, um die Clusterermittlung mit einer hohen GPU-Rechenkapazität zu
beschleunigen.

    \section{CoreSets Erzeugen}

CoreSets sind Räume mit einer Anhäufung mehrerer Punkte. Sie werden durch die frei wählbaren Parameter des Algorithmus, die vom
DBSCAN-Algorithmus inspiriert wurden \cite{7022654}, berechnet: $\epsilon$ und $\tau$. Ein
Referenzpunkt $P_{i}$ in
einem Subspace S ist genau dann mit einem anderen Punkt $P_{j}$ in S benachbart ($N^{S}(P_{i})$),
wenn $dist(P_{i}^{S}, P_{j}^{S})$ < $\epsilon$ und $P_{i} \neq P_{j}$. Außerdem muss gelten,
dass
$|N^{S}(P_{i})| >= \tau$. Ein CoreSet besteht also aus mindestens $\tau$ Punkten, die jeweils maximal $\epsilon$ voneinander entfernt sind.
CoreSets können Schnittpunkte in gemeinsamen Punkten bilden.

\begin{figure}[h]
    \centering
    \includegraphics[width=0.5\textwidth]{SUBSCALE_nachbarschaft}
    \caption[Corset Erzeugung]{CoreSet Erzeugung}
    \label{img:CoresetsErzeugen}
\end{figure}


Auf diese Weise werden für jede Dimension CoreSets gebildet.
    \section{Analyse der vorhandenen Implementierung}
Zur Verfügung steht eine Java- sowie eine C-Implementierung des Algorithmus. Die Implementierung in C wurde näher
betrachtet, da diese die Möglichkeit bietet, Algorithmusschritte mithilfe der Grafikkarte zu beschleunigen.

\subsection{Aufbau des Programmes}
Das Programm lässt sich mit einer JSON-Konfigurationsdatei, welche im ersten Schritt eingelesen und deserialisiert wird,
parametrisieren. Hier werden beispielsweise Algorithmushyperparameter wie $\tau$ und $\epsilon$ bestimmt oder die
Berechnung über die Grafikkarte aktiviert bzw. deaktiviert.\\
Der Datensatz wird als CSV-Datei bereitgestellt, welche in einen Vektor mit Elementen einer internen Datenstruktur
„DataPoint“ umgewandelt wird. DataPoint ist hier irreführend, da die Datenstruktur eine Dimension und die Werte dieser
Dimension aller Datensätze hält.\\
Mit Nutzung des Strategy-Patterns wird abhängig der Konfiguration die sequenzielle oder die parallele Subscale Strategie
geladen.\\
Die Methode „calculateClusterCandidates“, die beide Strategien zur Verfügung stellen, durchläuft den eingangs
beschriebenen Algorithmus und liefert die Subspaces des übergebenen Datensatzes. Aus den Subspaces werden im Anschluss
Cluster mittels DBSCAN ermittelt. Die gefundenen Cluster werden in einer CSV-Datei persistiert.

\subsection{Sequenzielle Implementierung}
Die sequenzielle Strategie instanziiert factory-ähnlich Abarbeitungsklassen, welche als Referenz an die generische
Methode „calculateAllSlices“ übergeben werden.\\
Die Methode unterteilt den Gesamtraum in disjunkte Teilräume. Für jeden Teilraum werden sequenziell die Dense Units
bestimmt, gefiltert und zu Subspaces geclustert um diese Zwischenergebnisse anschließend pro Slice in einer eigenen
CSV-Datei abspeichern zu können.\\
Im Anschluss werden die Dateien mit Zwischenergebnissen Datei für Datei zu einem Endergebnis kombiniert. Das
Gesamtergebnis wird dem DBSCAN-Algorithmus übergeben. Die finalen Cluster werden in eine Datei geschrieben.

\subsection{GPU-beschleunigte Implementierung}
Die GPU beschleunigte Implementierung unterscheidet sich von der sequenziellen Implementierung bei der Instanziierung
der Abarbeitungsklassen. „DenseUnitCreator“ und „SubspaceJoiner“ ersetzen die CPU basierten Berechnungen durch
Berechnungen auf der Grafikkarte.\\
Das Zusammenführe der Teilergebnisse erfolgt mit der identischen Logik wie im sequenziellen Algorithmus.

\section{Restrukturierung in einer neuen Anwendung}
Mit einer architektonischen Umstrukturierung sollen verschiedene Phasen des Algorithmus beliebig austauschbar werden,
sodass verschiedene Verteil- und Beschleunigungsverfahren miteinander verglichen werden können.\\
Subscale lässt sich in die Phasen Import, CoreSet-Seeker, DenseUnit-Generation, Subspace-Detection, Subspace-Combination
und Export unterteilen. Jede der Phasen wird mit einem Interface abstrahiert. Für ein Proof of Conecpt soll der Subscale
Algorithmus in einer sequenziellen Ausführung mit der Neustrukturierung implementiert werden.

\subsection{Import}
Die Importer implementieren eine Methode, die Tensordaten aus einem Medium auslesen und diese als Kollektion der
Datenstruktur „Dimension“ bereitstellen. Die Dimension hält ihre Größe, sowie ein Set von Punkten.
\begin{figure}[h]
	\centering
	\includegraphics[width=0.5\textwidth]{./Bilder/Restrukturierung/Importer.png}
	\caption{ImporterInterface}
\end{figure}
Für das Projekt konkret implementiert wurde ein CsvImporter, welcher die Daten aus einer CSV-Datei ausließt und in
Dimensions zur weiteren Verarbeitung umwandelt.\\
Der Datenimport kann hier mit weiteren konkreten Implementierungen beliebig ergänzt oder ersetzt werden. Denkbar wäre
beispielsweise der Import anderer Datenformate wie JSON oder XML oder das Lesen von entfernt bereitgestellten Ressource
wie beispielsweise eine CSV-Datei auf einem SFTP-Server.

\subsection{CoreSet-Seeker}
Der CoreSet-Seeker iteriert durch die übergebene Dimension und liefert für diese alle gefundenen CoreSets.
\begin{figure}[h]
	\centering
	\includegraphics[width=0.5\textwidth]{./Bilder/Restrukturierung/CoreSetSeeker.png}
	\caption{CoreSetSeekerInterface}
\end{figure}
Konkret implementiert wurde ein SequentialCoreSetGenerator. Dieser iteriert mit zwei geschachtelten for-Schleifen über
alle Punkte der übergebenen Dimension und berechnet hiermit die euklidische Distanz zwischen allen Punkten.
Unterschreitet diese $\epsilon$, werden die Punkte zu einem CoreSet zusammengefasst.\\
Weiter denkbar wäre die Implementierungen ParallelCoreSetGenerator, welcher die CoreSets auf einer GPU berechnen lässt,
indem mehreren Threads unterschiedliche Dimensionen oder gar einzelne Punkte einer Dimension zugewiesen werden. Die
Resultate werden im Anschluss gesammelt und zusammengeführt.\\
Auch eine Implementierung DistributeCoreSetGenerator könnte den Algorithmus beschleunigen. Diese Implementierung
dekoriert eine der vorherigen Implementierungen, indem diese den Aufruf mittels GRPC an entfernte Maschinen
weiterleitet, auf dessen Berechnungsergebnisse wartet um diese dann zusammengeführt zurückgeben zu können.

\subsection{Dense Unit Generator}
Der Dense Unit Generator erstellt für jedes CoreSet alle Kombinationen der Größe $\tau$. Diese Kombinationen werden in
der Datenstruktur DenseUnit abgelegt. DenseUnits werden gesammelt und als Collection zurückgegeben.
\begin{figure}[h]
	\centering
	\includegraphics[width=0.5\textwidth]{./Bilder/Restrukturierung/DenseUnitGenerator.png}
	\caption{DenseUnitGeneratorInterface}
\end{figure}
Die Implementierung SequentialDenseUnitGenerator übernimmt genau dies in for-Schleifen. Zukünftige Implementierungen
ParallelDenseUnitGenerator und DistributedDenseUnitGenerator parallelisieren die Generierung der Dense Units zum
Beispiel durch Zuweisung eines CoreSets pro Thread und verteilen diese.

\subsection{Subspace Detector}
Der Subspace Detector untersucht Dense Units auf Kollisionen und findet dadurch Subspaces: Anhäufungen vom Punkten über
mehrere Dimensionen.
\begin{figure}[h]
	\centering
	\includegraphics[width=0.5\textwidth]{./Bilder/Restrukturierung/SubspaceDetector.png}
	\caption{SubspaceDetectorInterface}
\end{figure}
Der SequentialSubspaceDetector iteriert über alle DenseUnits und prüft deren Signatur auf Kollision mit anderen
DenseUnits. Kollidieren mehrere Dense Units, bilden diese einen Subspace. Die Methode gibt alle entdeckten Subspaces
zurück.\\
Ein ParallelSubspaceDetector und ein DistributedSubspaceDetector könnten die gleiche Aufgabe durch Verteilen der zu
untersuchenden DenseUnits auf Threads und durch Verteilen auf entfernte Maschinen und anschließendes Zusammenführen
erledigen.

\subsection{Subspace Combiner}
Der Subspace Combiner erstellt aus den Subspaces, welche mehrere Dense Units bündeln, Cluster.
\begin{figure}[h]
	\centering
	\includegraphics[width=0.5\textwidth]{./Bilder/Restrukturierung/SubspaceCombiner.png}
	\caption{SubspaceCombinerInterface}
\end{figure}
Der SequentialSubspaceCombiner iteriert über alle Punkte der im Subspace vorkommenden Dense Units und eliminiert
Duplikate beim Zusammenfassen zu Cluster. Dies geschieht durch Iteration über alle Subspaces, deren Dense Units und den
dort enthaltenen Punkten.\\
Der Bau der Cluster funktioniert für jeden Subspace unabhängig, weswegen die Subspacekombination in einem
ParallelSubspaceCombiner und einem DistributedSubspaceCombiner parallelisiert und verteilt werden kann.

\subsection{Factory}
Die Definition der Schnittstellen erlaubt eine beliebige Konfiguration des Algorithmus. Eine Subscale Klasse wird mit
der Übergabe eines CoreSetSeekers, eines DenseUnitGenerators, eines SubspaceDetectors und eines SubspaceCombiners
instanziiert. Deren Methode „getClusters“ nimmt ein Set von Dimensionen, iteriert über diese um mit dem CoreSetSeeker
alle CoreSets zu erhalten, bildet aus den CoreSets Dense Units, in welchen anschließend Subspaces gesucht werden, die
die Methode schlussendlich zu Cluster kombiniert. DBSCAN übernimmt das finale Clustering der durch Subscale gefundenen
Clusterkandidaten.
\begin{figure}[h]
	\centering
	\includegraphics[width=0.9\textwidth]{./Bilder/Restrukturierung/Factory.png}
	\caption{Factory}
\end{figure}
Da diese Architektur dem Open-Close-Principle folgt, bietet sie die Möglichkeit, dass mehrere Entwickler zeitgleich und
konfliktfrei verschiedene Module entwickeln können. Außerdem erlaubt sie eine feingranulare Analyse über
Optimierungsoptionen. So kann der Algorithmus beispielsweise aus einen verteilten CoreSetSeeker, einem parallelen
DenseUnitGenerator, einem sequenziellen SubspaceDetector und einem parallelen SubspaceCombiner bestehen. Für beliebige
Kombinationen kann die Ausführungszeit gemessen werden, um zu analysieren, bei welchen Prozessschritten durch die
Parallelisierung und Verteilung tatsächlich ein Zeitgewinn zu verzeichnen ist.

\subsection{Probleme}
Da parallel die Anpassung der Codevorlage analysiert wurde, entschieden wir uns für ein „Proof of Concept“. Nach dessen
Umsetzung wollen wir abwägen, welche Lösung für die Projektumsetzung herangezogen wird.\\
Das Proof of Concept umfasste die dargelegte Architektur und die konkrete Implementierung für die sequenzielle
Abarbeitung des Algorithmus. Der SequentialCoreSetSeeker, der SequentialDenseUnitGenerator, der
SequentialSubspaceDetector und der SequentialSubspaceCombiner wurden implementiert und in einer
SequentialSubscaleFactory kombiniert. Ebenfalls implementiert wurde ein CSV-Importer zum Einlesen des Testdatensatzen.\\
Für kleine Datensätze war die Umsetzung lauffähig und lieferte Ergebnisse. Allerdings war die Laufzeit um ein Vielfaches
länger. Größere Datensätze konnten aufgrund eines Speicherüberlaufs nicht bearbeitet werden. Bei der Implementierung
wurde das von C++ benötigte akribische Speichermanagement zu sehr vernachlässigt was die Performanz der Anwendung massiv
einschränkte.\\
Des Weiteren entschieden wir uns nach der PoC-Phase gegen die Fortsetzung dieser Lösung, da die Cuda-Codeabschnitte aus
der Codevorlage nicht einfach übernommen werden konnten. Für Anpassungen im Cuda-Code fehlt im Team die Expertise. Eine
parallele Umsetzung ist folglich nicht möglich.

    \section{Kollision von Dense Units}

Da die Dense Units auch in unterschiedlichen Dimensionen existieren können, müssen diese als solche gekennzeichnet werden. In diesem Schritt ist das Subspace Clustering erkennbar. Direkt nach Berechnung der Dense Units, werden diese mit anderen Dense Units verglichen, um höhere Subspaces zu finden. Um Kollisionen in den Dense Units festzustellen, muss jedem Punkt im Datenset eine hohen Zufallszahl zugewiesen werden. Dieser Schritt wird vor dem Ausführen des SUBSCALE Algorithmus durchgeführt. Für alle Dense Units werden Signaturen gebildet. Diese Signatur ist die Summe aller Punkte in der Dense Unit. Anhand dieser Signatur werden die Dense Units in eine Tabelle eingetragen, mit den dazugehörigen Punkten und der Dimension. Bei einer Kollision der Signaturen, wird die Dimension zu dem bereits vorhanden Eintrag hinzugefügt (siehe Abbildung \ref{dense-collision}). \cite{Ramin}

\begin{figure}[h]
	\centering
	\includegraphics[width=0.5\textwidth]{DenseUnitKollision.png}
	\caption{Kollisionsauflösung von Dense Units}
	\label{dense-collision}
\end{figure}



    \section{Abbildung Dense-Units auf Subspaces}\label{sec:chap5}

Für das endgültige Clustering muss die Ausgabe des SUBSCALE-Algorithmus in die Struktur von den
Clusterkandidaten abgebildet werden. Jeder Kandidat besteht aus Dimensionen und eindeutigen Punkt
IDs. Dabei werden Punkte ausgesucht, welche in mehreren Dense-Units vertreten sind. Diese werden
dann zusammen innerhalb eines Subspaces definiert, wobei mehrfach vorkommende Punkte nur einmal
eingetragen werden. Die hier zusammengeführten Punkte haben die Eigenschaft, dass sie ziemlich
wahrscheinlich auch in dem jeweiligen Unterraum geclustert sind. Diese Eigenschaft macht diese
Punkte zu günstigen Kandidaten für das endgültige Clustering, welches im folgenden Unterkaptiel
\ref{sec:chap6} näher beschrieben wird.

    \section{Cluster}
Für das Testen der neu implementierten Verteilung wird das Projekt auf einem Cluster ausgeführt. Dazu werden von der
Hochschule einige Rechner zur Verfügung gestellt.

\subsection{Zugriff}
Um generell auf das Cluster zugreifen zu können, werden n Personen benötigt. Dies liegt daran, dass das Cluster über
bwLehrPool-Remot erreichbar ist. Ein anderes Cluster konnte uns zu dem Zeitpunkt nicht zur Verfügung gestellt werden. Da
wir genug Personen im Team waren, um eine einigermaßen aussagekräftige Anzahl von Rechnern bereitstellen zu können, war
dies für uns kein Problem. Jeder der n Personen meldet sich dann auf einem der Rechner in dem IMLA Pool an. Dort können
dann die Container, welche später genauer erläutert werden, gestartet werden. Da diese alle in einem Netzwerk verfügbar
sind, entsteht keine Schwierigkeit eine Verbindung zwischen den Rechnern des Pools herzustellen.

\subsection{Dockerfile}
Um ein Server oder Client auf einem Cluster-Rechner zu starten, wird dieser innerhalb eines Docker-Containers gestartet.
Dieser Gedanke entstand aufgrund der Gegebenheiten, dass nicht immer alle benötigten Tools auf einem System installiert
sind. Durch das Benutzen eines Docker-Containers können auf diesen, sofern für diesen alle benötigten Tools installiert
sind, zum Beispiel die \verb|nvidia-container-runtime|, alle noch nicht vorhandenden aber benötigten Tools
nachinstalliert werden. In dem folgenden Listing \ref{code:Dockerfile} ist dieser Ansatz zu erkennen.

\begin{lstlisting}[caption=Dockerfile, label={code:Dockerfile}, captionpos=b]
    FROM nvcr.io/nvidia/cuda:12.0.0-devel-ubuntu20.04

    WORKDIR /subscale
    
    RUN apt update
    RUN DEBIAN_FRONTEND=nointeractive TZ=Etc/UTC apt install -y --allow-unauthenticated --no-install-recommends tzdata
    RUN apt install -y --allow-unauthenticated --no-install-recommends git curl zip unzip tar pkg-config gfortran python3 cmake nano
    
    COPY . .
    
    RUN make init
    RUN make install-dependencies
    RUN make build
    RUN make compile

\end{lstlisting}

CUDA-Images gibt es in drei Varianten und sind über das öffentliche NVIDIA Hub Repository verfügbar.

\begin{itemize}
    \item base: enthält ab CUDA 9.0 das absolute Minimum (libcudart) für den Einsatz einer vorgefertigten CUDA Anwendung.
    \item runtime: erweitert das Basis-Image um alle gemeinsam genutzten Bibliotheken des CUDA-Toolkits.
    \item devel: erweitert das Runtime-Image, indem es die Compiler-Toolchain, die Debugging-Tools, die Header und die statischen Bibliotheken hinzufügt.
\end{itemize}

Um bereits ein System zu haben, welches mit CUDA kompatibel ist, wurde sich für die \verb|devel|-Version entschieden.
Dabei wurde auch die neuste Version benutzt. Schlussendlich ist in Zeile 1 das verwendete Docker-CUDA-Image
``nvcr.io/nvidia/cuda:12.0.0-devel-ubuntu20.04'', für welches sich entschieden wurde, zu sehen. Aufgrund einer
Empfehlung von Prof. Dr.-Ing. Janis Keuper wird der Container direkt von NVIDIA genommen und nicht von Docker Hub.

Leider ist auch auf dem CUDA-Image nicht alles installiert, was benötigt wird, um das Projekt zu kompilieren. Aus diesem
Grund werden von Zeile 5 bis Zeile 7 alle notwendigen Tools installiert. Dabei gab es vor allem die Schwierigkeit,
\verb|cmake| zu installieren. Als Lösung dafür musste, getrennt von allem, der Befehl in Zeile 5
\verb|RUN apt update| und anschließend in Zeile 6 \verb|tzdata| installiert werden. Dies ermöglicht die Umgehung des
manuellen Setzen der Zeitzone, welche beim normalen Installieren von \verb|cmake| auftritt. Auch war es notwendig, das
\verb|DEBIAN_FRONTEND| auf \verb|nointeractice| zu setzen.

In Zeile 9 ist durch \verb|COPY . .| der Befehl zu erkennen, welcher das Projekt in den Container kopiert. Anschließend
werden von Zeile 11 bis 14 die normalen Befehle, welche benötigt werden, um den Code zu kompilieren, ausgeführt. Dafür
wird das zuvor beschriebene Makefile verwendet, was an dieser Stelle eine sehr große Vereinfachung darstellt.

In dem Dockerfile wird kein \verb|EXPOSE| festgelegt, um die spätere Containererstellung zu vereinfachen. Falls man auf
einem Rechner testweise mehrere Container starten möchte, sollten nicht dieselben Ports exposed werden. Auch kann so
sehr einfach automatisiert werden, welcher Container mit welchem Port laufen soll. Im späteren Verlauf des Dokuments
wird aufgezeigt, wie der Port bei dem \verb|docker run|-Befehl freigegeben wird.

\subsection{Docker-Ausführung}

Um den Container mit allen Konfigurationen zu bauen, wird das oben genannte Dockerfile verwendet. Dabei wird mit dem
Befehl
\begin{itemize}
    \item[] \verb|docker build -t subscale .|
\end{itemize}
der Container gebaut. Der Befehl wird im gleichen Verzeichnis wie das Dockerfile ausgeführt. Dies kann einige Zeit in
Anspruch nehmen, da dabei zuerst der Container heruntergeladen (falls noch nicht bereits geschehen) wird sowie das
gesamte Projekt und auch Sub-Module kompiliert werden. Vor allem das \verb|mlpack| scheint eine sehr zeitaufwendige
Kompilierung zu haben. Anschließend kann mit dem Befehl
\begin{itemize}
    \item[] \verb|docker run --name subscaleGPUServer --gpus all --rm -it -p 8080:8080|
    \item[] \verb|subscale make start-server p=8080|
\end{itemize}
der Container gestartet werden. Durch die Namensgebung mit \verb|--name| ist im anschließenden Schritt die Handhabung
mit dem laufenden Container einfacher. Auch muss mit \verb|--gpus all| der Zugriff auf GPU-Ressourcen ermöglicht werden.
Die Flag \verb|--rm| ist optional, aber empfohlen, da dies nach Beenden des Containers diesen automatisch entfernt.
Anzumerken ist, dass dabei nicht das Image, welches durch den \verb|docker build| Befehl entstanden ist, gelöscht wird.
Durch \verb|-p 8080:8080| wird der Port 8080 des Containers auf 8080 nach außen gemappt. Dann kann dieser Port dem
\verb|make start-server| als \verb|p=8080| übergeben werden, woraufhin dieser direkt mit dem Port 8080 gestartet wird.
Durch die Flag \verb|-it| bekommt man die Sicht auf den Server, wodurch zu erkennen ist, ob ein Request des Clients
ankam oder nicht.

Den Client, welcher die Berechnung startet, sollte nun über den folgenden Befehl ausgeführt werden.
\begin{itemize}
    \item[] \verb|docker run --name subscaleGPUClient --gpus all --rm -it|
    \item[] \verb|subscale /bin/bash|
\end{itemize}
Dadurch wird eine Konsole innerhalb des Containers geöffnet. Auch hier ist die Nutzung von \verb|-it| sinnvoll. Dies
liegt vor allem daran, dass dort auch das Ergebnis gespeichert wird. Anschließend kann durch \verb|make start-client|
der Prozess gestartet werden. Dafür muss natürlich die \verb|Client/config.json| richtig angepasst sein.

Generell erhält man die IP-Adresse eines laufenden Containers über den folgenden Befehl.
\begin{itemize}
    \item[] \verb|docker container inspect subscaleGPUServer || \verb| grep -i IPAddress|
\end{itemize}
Hierbei wird sichtbar, dass die Namensgebung den Vorteil bietet, nicht erst die ID des Containers herausfinden zu
müssen. So ist es auch an dieser Stelle einfacher, eine Automatisierung zu ermöglichen.

    \section{Benchmarking}
Für das Benchmarking wurde von jedem Teammitglied jeweils ein Docker-Container mit Server gestartet. Anschließend wurde
auf einem der Cluster-Rechner der Client gestartet und die Ausführung gemessen. Das bedeutet, dass die gesamte
Ausführungszeit des Clients das Messergebnis ergibt. Bei der sequentiellen Messung wird ebenfalls innerhalb des
Containers der Subscale-Algorithmus gestartet, jedoch nicht die verteilte Version.

\subsection{Sequentiell}
Die sequentielle Berechnung innerhalb eines Containers ergab die folgende Ausführungszeit.

\begin{center}
    \begin{tabular}{ |c|c| }
        \hline
        Ausführungsmessung & Zeit    \\
        \hline
        1                  & 2.685 s \\
        2                  & 2.659 s \\
        3                  & 2.675 s \\
        4                  & 2.681 s \\
        5                  & 2.711 s \\
        6                  & 2.657 s \\
        7                  & 2.679 s \\
        8                  & 2.662 s \\
        9                  & 2.667 s \\
        10                 & 2.676 s \\
        \hline
        Durchschnitt       & 2.675 s \\
        \hline
    \end{tabular}
\end{center}

Hierbei ist auffällig, dass sich alle Messungen sehr nahe beieinander Befinden und davon auszugehen ist, dass der
gemessene Wert eine gute Basis für einen Vergleich darstellt.

\subsection{Verteilt}

Bei der verteilten Berechnung über mehrere Container und Rechner hinweg ergaben sich folgende Messungen.

\begin{center}
    \begin{tabular}{ |c|c| }
        \hline
        Ausführungsmessung & Zeit     \\
        \hline
        1                  & 10.736 s \\
        2                  & 11.285 s \\
        3                  & 10.982 s \\
        4                  & 10.924 s \\
        5                  & 10.859 s \\
        6                  & 10.908 s \\
        7                  & 11.753 s \\
        8                  & 11.813 s \\
        9                  & 11.348 s \\
        10                 & 11.633 s \\
        \hline
        Durchschnitt       & 11.224 s \\
        \hline
    \end{tabular}
\end{center}

Hier gibt es Abweichungen um bis zu einer Sekunde. Eine mögliche Quelle dieser Abweichungen besteht darin, dass das
Netzwerk nicht immer gleich auf Anfrage und Antwort reagieren kann, was auf die nicht alleinige Nutzung dieses Netzwerks
zurückzuführen ist.

\subsection{Vergleich}

Bei dem Vergleich der durchschnittlichen Laufzeit ist zu erkennen, dass der verteilte Algorithmus langsamer ist als der
sequenzielle. In der folgenden Tabelle nochmal die langsamste und schnellste Messung, als auch der Durchschnitt aller
Messungen gegenübergestellt.

\begin{center}
    \begin{tabular}{ |c|c|c| }
        \hline
        Typ                        & Sequentiell & Verteilt \\
        \hline
        langsamste Messung         & 2.711 s     & 11.813 s \\
        schnellste Messung         & 2.657 s     & 10.736 s \\
        duschschnitt der Messungen & 2.675 s     & 11.224 s \\
        \hline
    \end{tabular}
\end{center}

Die Vermutung liegt nahe, dass der uns zur Verfügung gestellten Testdatensatz nur bedingt für eine effizientere
Ausführung durch eine Verteilung geeignet ist, da dieser dafür potenziell zu klein ist. Bei einem Datensatz, welcher
noch sehr viel mehr Daten enthält, könnte sich der Overhead, welcher durch den Netzwerkverkehr entsteht, im Verhältnis
zu dem lokalen Berechnen wieder lohnen. Jedoch muss dies in einem folgenden Versuch überprüft werden, wenn auch solch
ein Datensatz zur Verfügung steht.


%%% Angabe der .bib-Datei (ohne Endung) / State .bib file (for BibTeX usage)
%\bibliography{Literatur}
%\printbibliography %if you use biblatex/Biber
\end{document}