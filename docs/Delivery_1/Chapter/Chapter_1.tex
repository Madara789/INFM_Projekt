\section{Daten Aufbereitung}

Der Subscale Algorithmus beginnt damit die Daten aufzubereiten. Dazu werden die einzelnen Punkte,
die in jeder Dimension enthalten sind, mit einem eindeutigen Index versehen. Die Idee hinter dem
Index besteht daraus, dass jeder Punkt eine eindeutige, hohe, zufällig gewählte Ganzzahl als
Schlüssel erhält. Später werden die Punkte zu Partitionen zusammengefasst. Dabei bildet die Summe
der Schlüssel die Signatur ab. Da jeder Schlüssel einen hohen Wert hat, besitzt die Summe der
Schlüssel ebenfalls einen hohen Wert. Laut \cite{7022654} wird für eine sehr hohe Ganzzahl, bei
sehr kleiner
Partitionsgröße, die Anzahl der einzigartigen, Partitionen bestimmter Größe astronomisch hoch.
Dadurch ist die Wahrscheinlichkeit, dass zwei Partitionen die gleiche Zahl als Signatur bilden
sehr gering.

Die Signaturen werden verwendet, um paarweise identische \emph{Dense Units} (siehe
\ref{sec:chap3}) $U_1^{d_a}$, $U_2^{d_b}$ zwischen den Dimensionen $d_a$, $d_b$ zu ermitteln.


\section{Daten Projektion}

Ein Datensatz besteht aus \emph{n} Zeilen $\cdot$ \emph{k} Spalten, wobei eine Zeile jeweils ein Punkt
$P^{k}$ als k-Dimensionaler Vektor in \emph{k} Spalten repräsentiert. Somit besteht der gesamte
Datensatz aus $P_{n}$ Punkten.
Der Datensatz wird zu einem Subspace S der Größe einer Dimension projeziert, sodass sämtliche
Punkte in einer Dimension verglichen werden können. Für die Nachbarschaftsbeziehung kann als
Distanzmaß z.B. die euklidische Distanz angenommen werden.
