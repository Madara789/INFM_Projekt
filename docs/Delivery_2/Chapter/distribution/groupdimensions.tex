\subsection{Gruppierung nach Dimensionen}

Die in dem vorherigen Schritt erstellte Tabelle, bestehend aus Signatur, Punkten und Dimensionen, wird nach Erstellen dieser, anhand der Spalte \verb|Dimensionen| gruppiert. In diesem Schritt wäre es möglich, die Arbeit aufzuteilen und jeder Teilnehmer schaut in seinem zuständigen Tabellenbereich nach Überschneidungen und gruppiert diese. Anschließend muss jedoch das Ergebnis aller erneut auf Überschneidungen geprüft und gegebenenfalls erneut gruppiert werden. Je nach Masse der Daten könnte dies erneut verteilt werden, bevor eine Instanz die endgültige Berechnung vornimmt. Der Vorteil hierfür wäre, eine große Masse verteilt zu ermitteln und abschließend nur einen kleinen Teil für das Endergebnis berechnen zu müssen. Aber durch das erneute Prüfen der verteilt berechneten Zwischenergebnissen, entsteht mehr Aufwand als durch das einmalige iterieren über die Tabelle. Demzufolge muss überprüft werden, ob durch das Verteilen und wieder Zusammenfügen der Daten, wirklich ein Geschwindigkeitsvorteil gegenüber der direkten Berechnung auf einem Rechner entsteht oder ob dieser vernachlässigbar gering ist.
