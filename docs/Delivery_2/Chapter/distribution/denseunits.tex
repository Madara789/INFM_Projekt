\subsection{DenseUnits}

Eine Dense Unit beschreibt die Menge aller Kombinationen von Punkten in einem Core Set als Tupel mit der Größe der
minimalen Core Set Größe. Die Berechnung der Dense Units ist für jedes Core Set unabhängig.\\
Sind die Core Sets ermittelt, so können diese an unabhängige Recheneinheiten distributiert werden. Die Recheneinheit
erstellt dann alle Tupel des Core Sets und persistiert diese in einem geeigneten Medium.\\
Die Dense Unit Bildung selbst kann ebenfalls weiter parallelisiert werden. Die Kombination der im Core Set enthaltenen
Punkte erfolgt sequentiell nach Schema: Bei einer minimalen Core Set Größe von drei Punkten wird für die Bildung der
Dense Units zunächst der Punkt 1 und 2 mit dem Punkt 3, dann 4, dann 5 usw. kombiniert: (1, 2, 3), (1, 2, 4), (1, 2, 5), ...
Sind alle Kombinationen, die die Punkte 1 und 2 enthalten ermittelt, folgen alle Kombinationen, die die Punkte 1 und 3 enthalten:
(1, 3, 4), (1, 3, 5), ... Das Intervall kann auf mehrere Recheneinheiten verteilt werden. Die Recheneinheiten ermitteln dann
jeweils die Dense Units aus ihrem Intervall und legen diese im gemeinsamen Speicher ab.