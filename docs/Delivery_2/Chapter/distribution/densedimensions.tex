\subsection{System vergleicht Dense Units von Zwei Dimensionen}

Nachdem die Dense Units für eine Dimension und deren CoreSets erzeugt wurden, müssen für diese Dense Units Signaturen gebildet werden und die Punkte mit deren Dimension in eine Tabelle eingetragen werden. Die Berechnung der Signatur ist unabhängig von anderen Dimensionen und Dense Units, somit kann dies auf verschiedene Rechnern Verteilt werden. Das eintragen der Daten muss jedoch über Rechnergrenzen hinweg synchronisiert werden. Wenn ein Rechner für die übermittelte Dense Unit die Signatur erzeugt, muss dieser vorerst prüfen ob bereits ein Eintrag mit dieser Signatur existiert und ergänzt die Dimension an diesem Eintrag. Wenn beispielsweise für die Datenhaltung eine SQL oder NoSQL Datenbank verwendet wird, muss bevor ein Eintrag gespeichert wird, zuerst abgefragt werden ob die Signatur bereits hinterlegt ist. Bei großen Datenmengen erzeugt das ständige abfragen ob die Signatur bereits existiert einen großen overhead beim speichern. Eine mögliche Alternative um diese Abfrage zu umgehen, kann eine eigene Implementierung einer Distributed Hash Table verwendet werden. Dabei werden die Dense Units sowie die Dimension dem Peer-to-Peer Netzwerk übergeben, dieses berechnet aus den Punkten die Summe (Hashwert) der Dense Unit und beauftragt den zuständigen Knoten die Daten zu speichern. Der Zuständige Knoten müsste bei einer Kollision der Signatur die Dimension am Eintrag ergänzen. Dadurch würde das berechnen der Signatur auf dem Peer-to-Peer Netzwerk verteilt werden und für Kollisionen ist nur der Zuständige Knoten verantwortlich.