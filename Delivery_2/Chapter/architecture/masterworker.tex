\subsection{Master-Worker-Architektur}

Viele Algoritmusschritte von Subscale können teilweise vollständig parallelisiert werden. So is es möglich, Subscale über
ein Datenset in Arbeitspakete aufzuteilen, die dann unabhängig bearbeitet werden können. Die Arbeitspakete werden in einer
gemeinsamen Queue abgelegt. Zum System können sich beliebig viele Worker zuschalten. Die Worker verfügen über einen Client,
der die Kommunikationstechnologie kennt, Arbeitspakete von der Warteschlange konsumieren und je nach Arbeitspakettyp intern
routen kann. Für jeden Arbeitsschritt existiert eine eigene Anwendung. Die Anwendung konsumiert und entpackt das Arbeitspaket,
erledigt die Aufgabe, verpackt das Resultat in einem neuen Arbeitspaket und platziert dieses wieder in der Queue. Zwischenergebnisse
können unterstützend in einem geeignet schnellen Persistierungslayer gespeichert werden.

Vorteile:
\begin{itemize}
    \item Das System lässt sich einfach auf beliebig viele Instanzen skalieren
    \item Der Algorithmus kann schrittweise parallelisiert werden
    \item Eine Benchmarkanalyse zur Performanzsteigerung ist durch den modularen Aufbau ohne großen Mehraufwand möglich
    \item Unterschiedliche Arbeitsschritte können mit unterschiedlichen und demzufolge den geeignetsten Technologien implementiert werden
    \item Der modulare Aufbau ermöglich eine konfliktarme parallele Implementierung
\end{itemize}

Nachteile:
\begin{itemize}
    \item Es ist nicht bekannt, in wie weit die Queueing-Technologie oder der Persistierungslayer ein Bottleneck bildet
    \item Die Fehleranalyse erfordert ein umfangreiches Logging
    \item Asynchrone Abarbeitungen erhöhen die Aggregations- und Koordinationskomplexität
\end{itemize}