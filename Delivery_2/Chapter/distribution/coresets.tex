\subsection{CoreSets}

Der Algorithmus beginnt damit, die CoreSets jeder Dimension zu bilden. Die Bildung der CoreSets
geschieht innerhalb einer Dimension und ist zudem auch komplett unabhängig von den anderen
Dimensionen. Dabei wird innerhalb einer Dimension jeder Punkt und die Entfernungen
dieses Punktes zu anderen Punkten betrachtet. Sollte die Entfernung kleiner als der angegebene Epsilon-Wert sein, so
wird der Punkt zum CoreSet hinzugefügt.

Da in diesem Prozess keine Abhängigkeit zu anderen Dimension besteht und lediglich
die Daten der jeweiligen Dimension benötigt werden, kann die Bildung der CoreSets für jede
Dimension parallel abgearbeitet werden.

Somit kann erreicht werden, dass jeder PC in einem Cluster nur ein Subset von Dimensionen
abarbeitet, um die Corsets zu bilden.