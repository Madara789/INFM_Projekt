\section{Einleitung}

Anders, als konventionelle Ansätze bei Clustering-Algorithmen, die sämtliche Features auf einmal
vergleichen \cite{7022654}, zielt der Subscale Algorithmus darauf ab, die hochdimensionalen Daten in Teilen
zu effizient verarbeiten.
Dabei möchte der Algorithmus das durch hohe Dimensionalität bedingte, so genannte Problem \emph{Curse
of Dimensionality} lösen, indem es die Abgeschlossenheit des Apriori-Prinzips
\cite{TechnicalReviewonBuprenorphine:anAlternativeTreatmentforOpioidDependence.1992} nutzt und die
Teilmengen der
Datensatzfeatures sukzessive, bottom-up aufbaut.

Das Apriori-Prinzips ermöglicht es aus der Gesamtmenge der Dimensionen in Teilmengen davon, den
so genannten Subspaces, so zu verarbeiten, dass anstatt \emph{$2^{k}-1$ möglichen Achsen-parallele
Subspaces in \emph{k} Dimensionen} \cite{7022654}, nur die benötigten Subspaces berechnet werden.
