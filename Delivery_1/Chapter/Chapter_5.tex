\section{Abbildung Dense-Units auf Subspaces}\label{sec:chap5}

Für das endgültige Clustering muss die Ausgabe des SUBSCALE-Algorithmus in die Struktur von den Clusterkandidaten abgebildet werden. Jeder Kandidat besteht aus Dimensionen und eindeutigen Punkt IDs. Dabei werden Punkte ausgesucht, welche in mehreren Dense-Units vertreten sind. Diese werden dann zusammen innerhalb eines Subspaces definiert, wobei mehrfach vorkommende Punkte nur einmal eingetragen werden. Die hier zusammengeführten Punkte haben die Eigenschaft, dass sie ziemlich wahrscheinlich auch in dem jeweiligen Unterraum geclustert sind. Diese Eigenschaft macht diese Punkte zu günstigen Kandidaten für das endgültige Clustering, welches im folgenden Unterkaptiel \ref{sec:chap6} näher beschrieben wird.
